\documentclass[]{article}

\usepackage[left=2.00cm, right=2.00cm, top=2.00cm, bottom=2.00cm]{geometry}
\usepackage[spanish,es-noshorthands]{babel}
\usepackage[utf8]{inputenc} % para tildes y ñ
\usepackage{graphicx} % para las figuras
\usepackage{xcolor}
\usepackage{listings} % para el código fuente en c++

\lstdefinestyle{customc}{
  belowcaptionskip=1\baselineskip,
  breaklines=true,
  frame=single,
  xleftmargin=\parindent,
  language=C++,
  showstringspaces=false,
  basicstyle=\footnotesize\ttfamily,
  keywordstyle=\bfseries\color{green!40!black},
  commentstyle=\itshape\color{gray!40!gray},
  identifierstyle=\color{black},
  stringstyle=\color{orange},
}
\lstset{style=customc}


%opening
\title{Práctica 2. Programación dinámica}
\author{Adrian Reyes Alba \\ % mantenga las dos barras al final de la línea y este comentario
adrian.reyesalba@alum.uca.es \\ % mantenga las dos barras al final de la línea y este comentario
Teléfono: 647243014 \\ % mantenga las dos barras al final de la linea y este comentario
NIF: 49074565V \\ % mantenga las dos barras al final de la línea y este comentario
}


\begin{document}

\maketitle

%\begin{abstract}
%\end{abstract}

% Ejemplo de ecuación a trozos
%
%$f(i,j)=\left\{ 
%  \begin{array}{lcr}
%      i + j & si & i < j \\ % caso 1
%      i + 7 & si & i = 1 \\ % caso 2
%      2 & si & i \geq j     % caso 3
%  \end{array}
%\right.$

\begin{enumerate}
\item Formalice a continuación y describa la función que asigna un determinado valor a cada uno de los tipos de defensas.

He seguido el algoritmo A*.
Tenemos dos vectores donde se van a guardar los nodos que esten abiertos(estará aqui hasta que no recorramos) o cerrados(cuando pasemos por él),
y inicializamos la celda inicial con los valores de G, H, F y el nodo padre que
será nulo e incluyéndolo en la lista de nodos abiertos.
Para la solución de la heuristica he usado la distancia euclídea.
nodos de la lista de nodos abiertos. El algoritmo sacará el nodo con menor coste y compararía si ese nodo es
el nodo destino, si es destino ya tendríamos solución y finalizaría, y encaso contrario miraríamos
lista de nodos adyacentes y calcularía sus valores(F, G, H, nodo padre). Estos nodos pasarían a la lista de
nodos abiertos. En el caso de que un nodo ya esté en la lista, se calcularía el coste del camino por el actual y
se compararía con el que tenía. Si es mayor no hace nada y si es menor se actualiza.
Una vez encontremos una solución correcta, o nos quedemos sin poder encontrarla procederemos a recuperar
el camino.


\item Describa la estructura o estructuras necesarias para representar la tabla de subproblemas resueltos.

Escriba aquí su respuesta al ejercicio 2.

\item En base a los dos ejercicios anteriores, diseñe un algoritmo que determine el máximo beneficio posible a obtener dada una combinación de defensas y \emph{ases} disponibles. Muestre a continuación el código relevante.

\begin{lstlisting}
/*PARA LA EXTRACTORA*/
	cellValueExtractora(cell, nCellsWidth,fil,col);
	while(currentCell != listCells.end() && flag == false) {
		selectCell(cell,nCellsWidth,(*currentCell));
		if(factible(mapWidth,mapHeight,obstacles,defenses,(*currentExtraction),(*currentCell),defensesPutted)) {

		    flag = true;
		    (*currentExtraction)->position.x = (*currentCell).position.x;
		    (*currentExtraction)->position.y = (*currentCell).position.y;
		    (*currentExtraction)->position.z = 0;
		   	fil = (*currentCell).id / nCellsWidth;
		   	col = (*currentCell).id % nCellsWidth;
		    defensesPutted.push_front(*currentExtraction);


		}

		++currentCell;
	}
\end{lstlisting}


\item Diseñe un algoritmo que recupere la combinación óptima de defensas a partir del contenido de la tabla de subproblemas resueltos. Muestre a continuación el código relevante.

1.Conjunto de candidatos: conjunto de celdas.
2.Conjunto de candidatos seleccionados: celdas seleccionadas
3.Función solución: Comprueba que esta colocada la base y las defensas.
4.Función de selección: Seleccionamos las casillas con mayor valor, es decir las que están más en el centro del mapa.
Función de factibilidad: si no sale del mapa, no choca con obstaculos y no choca con ninguna defensas.
Función objetivo: que la base este rodeada de defensas.
Objetivo: Que tarde lo menos posible en destruirse la extractora.


\end{enumerate}

Todo el material incluido en esta memoria y en los ficheros asociados es de mi autoría o ha sido facilitado por los profesores de la asignatura. Haciendo entrega de este documento confirmo que he leído la normativa de la asignatura, incluido el punto que respecta al uso de material no original.

\end{document}
