He seguido el algoritmo A*.
Tenemos dos vectores donde se van a guardar los nodos que esten abiertos(estará aqui hasta que no recorramos) o cerrados(cuando pasemos por él),
y inicializamos la celda inicial con los valores de G, H, F y el nodo padre que
será nulo e incluyéndolo en la lista de nodos abiertos.
Para la solución de la heuristica he usado la distancia euclídea.
nodos de la lista de nodos abiertos. El algoritmo sacará el nodo con menor coste y compararía si ese nodo es
el nodo destino, si es destino ya tendríamos solución y finalizaría, y encaso contrario miraríamos
lista de nodos adyacentes y calcularía sus valores(F, G, H, nodo padre). Estos nodos pasarían a la lista de
nodos abiertos. En el caso de que un nodo ya esté en la lista, se calcularía el coste del camino por el actual y
se compararía con el que tenía. Si es mayor no hace nada y si es menor se actualiza.
Una vez encontremos una solución correcta, o nos quedemos sin poder encontrarla procederemos a recuperar
el camino.
