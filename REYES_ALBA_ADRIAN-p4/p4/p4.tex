\documentclass[]{article}

\usepackage[left=2.00cm, right=2.00cm, top=2.00cm, bottom=2.00cm]{geometry}
\usepackage[spanish,es-noshorthands]{babel}
\usepackage[utf8]{inputenc} % para tildes y ñ

%opening
\title{Práctica 4. Exploración de grafos}
\author{Adrian Reyes Alba \\ % mantenga las dos barras al final de la línea y este comentario
adrian.reyesalba@alum.uca.es \\ % mantenga las dos barras al final de la línea y este comentario
Teléfono: 647243014 \\ % mantenga las dos barras al final de la linea y este comentario
NIF: 49074565V \\ % mantenga las dos barras al final de la línea y este comentario
}


\begin{document}

\maketitle

%\begin{abstract}
%\end{abstract}

% Ejemplo de ecuación a trozos
%
%$f(i,j)=\left\{ 
%  \begin{array}{lcr}
%      i + j & si & i < j \\ % caso 1
%      i + 7 & si & i = 1 \\ % caso 2
%      2 & si & i \geq j     % caso 3
%  \end{array}
%\right.$

\begin{enumerate}
\item Comente el funcionamiento del algoritmo y describa las estructuras necesarias para llevar a cabo su implementación.

He seguido el algoritmo A*.
Tenemos dos vectores donde se van a guardar los nodos que esten abiertos(estará aqui hasta que no recorramos) o cerrados(cuando pasemos por él),
y inicializamos la celda inicial con los valores de G, H, F y el nodo padre que
será nulo e incluyéndolo en la lista de nodos abiertos.
Para la solución de la heuristica he usado la distancia euclídea.
nodos de la lista de nodos abiertos. El algoritmo sacará el nodo con menor coste y compararía si ese nodo es
el nodo destino, si es destino ya tendríamos solución y finalizaría, y encaso contrario miraríamos
lista de nodos adyacentes y calcularía sus valores(F, G, H, nodo padre). Estos nodos pasarían a la lista de
nodos abiertos. En el caso de que un nodo ya esté en la lista, se calcularía el coste del camino por el actual y
se compararía con el que tenía. Si es mayor no hace nada y si es menor se actualiza.
Una vez encontremos una solución correcta, o nos quedemos sin poder encontrarla procederemos a recuperar
el camino.


\item Incluya a continuación el código fuente relevante del algoritmo.

Escriba aquí su respuesta al ejercicio 2.


\end{enumerate}

Todo el material incluido en esta memoria y en los ficheros asociados es de mi autoría o ha sido facilitado por los profesores de la asignatura. Haciendo entrega de esta práctica confirmo que he leído la normativa de la asignatura, incluido el punto que respecta al uso de material no original.

\end{document}
