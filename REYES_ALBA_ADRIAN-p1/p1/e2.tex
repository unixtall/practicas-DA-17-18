Lo que hago en la función de factibilidad es comprobar 3 casos. Primer caso: que se salga del mapa. Segundo caso: que choque con una defensa ya puesta. Y tercer caso: Que no choque con un obstaculo

\begin{lstlisting}
bool factible(float mapWidth,float mapHeight, std::list<Object*> obstacles, std::list<Defense*> defenses,
                Defense* defensa,Cell currentCell, List<Defense*> defensesPutted) {
	List<Defense*>::iterator currentDefense = defensesPutted.begin();
	List<Object*>::iterator currentObstacles = obstacles.begin();
	//Comprobamos que no se salga del mapa
	if((currentCell.position.x - defensa->radio) < 0 || (currentCell.position.x + defensa->radio  >= mapWidth)||(currentCell.position.y - defensa->radio) < 0 || (currentCell.position.y + defensa->radio ) >= mapHeight)
				return false;

	while(currentDefense != defensesPutted.end()){

		if(defensa->id != (*currentDefense)->id){

			if(_distance((*currentDefense)->position, currentCell.position) <= (*currentDefense)->radio + defensa->radio) //Comprobamos con las defensas

				return false;

		currentDefense++;
		}
	}

 while(currentObstacles != obstacles.end()){
		if(_distance((*currentObstacles)->position, currentCell.position) <= (*currentObstacles)->radio + defensa->radio) //Comprobamos con los obstaculos
			return false;
		currentObstacles++;
	}

	return true;
	
}
\end{lstlisting}
